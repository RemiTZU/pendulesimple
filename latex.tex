\documentclass[12pt]{article}

\usepackage{EngReport}

\graphicspath{{Images/}}
\bibliography{Sources}
\onehalfspacing
\graphicspath{{images/}}
\geometry{letterpaper, portrait, includeheadfoot=true, hmargin=1in, vmargin=1in}

%\fontsize{font size}{vertsize (usually 1.2x)}\selectfont

\begin{document}
\renewcommand{\familydefault}{\rmdefault}
\documentclass[12pt]{article}

\usepackage{EngReport}

\graphicspath{{Images/}}
\bibliography{Sources}
\onehalfspacing
\graphicspath{{images/}}
\geometry{letterpaper, portrait, includeheadfoot=true, hmargin=1in, vmargin=1in}

%\fontsize{font size}{vertsize (usually 1.2x)}\selectfont

\begin{document}
\renewcommand{\familydefault}{\rmdefault}

\input{titlepage}


\pagebreak

Le pendule simple est un système mécanique constitué d'une masse \(m\) suspendue à une tige de longueur \(L\), sans frottement. L'angle que forme la tige avec la verticale est représenté par \(\theta\) (en radians).

Pour trouver l'équation différentielle du mouvement du pendule simple, nous appliquons la seconde loi de Newton. La somme des forces agissant sur la masse est égale à sa masse multipliée par l'accélération :
\[ \Sigma \vec{F} = m \vec{a} \implies \vec{T}+\vec{P} = m \vec{a}\]


Les seules forces agissant sur la masse sont le poids \( P \) et la tension du fil \( T \).

Nous nous placerons dans un repère cartésien.
On peut ainsi définir deux équations selon l'axe 
\(Ox\) et \(Oy\) :
\[
\text{Ox : } mg - T_x \cos(\theta) = m\ddot{\a_x}
\]
\[
\text{Oy : } -\sin(\theta) = m \ddot{\a_y}
\]

par ailleurs on sait que 

\[
\begin{cases}
\vec{OM}_x = \cos(\theta) \cdot l \cdot \vec{u}_x \\
\vec{OM}_y = \cos(\theta) \cdot l \cdot \vec{u}_y
\end{cases}
\implies
\begin{cases}
\frac{d\vec{OM}_x}{dt} = -\[ \dot{\theta}\cdot \sin(\theta) \cdot l \cdot \vec{u}_x \\
\frac{d\vec{OM}_y}{dt} = \[ \dot{\theta} \cdot \cos(\theta) \cdot l \cdot \vec{u}_y \]
\end{cases}

\implies
\begin{cases}
\begin{aligned}
\frac{d^2\vec{OM}_x}{dt^2} &= -l\left(\ddot{\theta}\sin(\theta) + (\dot{\theta})^2 \cos(\theta)\right) \vec{u}_x \\
\frac{d^2\vec{OM}_y}{dt^2} &= l\left(\ddot{\theta}\cos(\theta) - (\dot{\theta})^2 \sin(\theta)\right) \vec{u}_y
\end{aligned}
\end{cases}

\quad



En multipliant la première équation par \(\sin(\theta)\) et la seconde par \(\cos(\theta)\) on obtient les équations suivantes :

\quad

\left\{
\begin{aligned}
 -T\cos(\theta)\sin(\theta) + mg\sin(\theta) = -ml\ddot{\theta}\sin^2(\theta) - ml(\dot{\theta})^2\cos(\theta)\sin(\theta) \\
 -T\cos(\theta)\sin(\theta) = ml\ddot{\theta}\cos^2(\theta) - ml(\dot{\theta})^2\sin(\theta)\cos(\theta)
\end{aligned}
\right.

\quad

Et en soustrayant (1) - (2) on peut trouver notre equation différentielle :

\[ mg\sin(\theta) = -ml\ddot{\theta} \quad \implies \quad \ddot{\theta} + \frac{g}{l}\sin(\theta) = 0 \]

\newpage
Passons à présent à l'approche numérique. Sachant que l'accélération est une variation de la vitesse par rapport au temps, on obtient l'égalité qui suit :
\[ v_{\theta}(t+\delta t) = \frac{d\theta}{dt}(t+\delta t) \approx v_{\theta}(t) + a_{\theta}(t) \times \delta t \]

\quad

avec  a_{\theta}(t) = \ddot{\theta} = -\frac{g}{l}\sin(\theta)

\quad

ainsi  \theta (t + \delta t) \approx \theta (t) + v_{\theta}(t+\delta t) \times \delta t

\quad

Dans l'approche numérique, 
(\(\delta t\)) représente un petit intervalle de temps utilisé pour effectuer les calculs itératifs. Cela nous permet d'approximer les dérivées temporelles en utilisant des pas de temps finis.

\quad

On peut à présent à chaque itération calculer \(v_{\theta}(t+\delta t)\) qui nous donnera par la suite \(\theta (t + \delta t)\).


\printbibliography

\end{document}
